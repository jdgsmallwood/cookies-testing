\documentclass{beamer}
\usetheme{Goettingen}

\title{Pytest: Where Bugs Come to Die, No Exceptions}
\author{Jay Smallwood}
\begin{document}
\begin{frame}[plain]
    \maketitle
\end{frame}
\begin{frame}
	\frametitle{Learning Intentions}
	\begin{itemize}
		\item Be able to explain when and why a piece of code needs tests.
		\item  Be able to structure tests using the Arrange-Act-Assert framework.
		\item  Be able to write and run simple tests using pytest \& ipytest.
		\item  Understand that writing testable code requires a slightly different method of writing code.
		\item Have all the necessary understanding to be able to start writing tests on your own code \textbf{today}.
	\end{itemize}
\end{frame}

\begin{frame}
	\frametitle{Structure}
	\begin{itemize}
		\item Setup local environments.
		\item  Examples of unit testing.
		\item  What is testing? Why do we write tests?
		\item  What code requires tests? What code does not require tests?
		\item What is Arrange-Act-Assert? 
		\item Refactoring code to make it testable.
		\item Public \& Private functions - what to test?
		\item Write a test on one of your own functions.
	\end{itemize}
	
\end{frame}

\section{Setup}
\begin{frame}
	\frametitle{Setting up the environment}
	\begin{itemize}
		\item Clone <insert git repo>
		\item Follow the instructions in the README.md
	\end{itemize}
	
\end{frame}

\section{Introduction}
\begin{frame}
	\frametitle{Some simple tests...}
	Refer to Example 1 in the notebook.
\end{frame}


\begin{frame}
	\frametitle{Why Test?}
	What do you think testing is? Why do we do it?
\end{frame}

\begin{frame}
	\frametitle{It feels like...}
	\includegraphics{images/wam.jpeg}
	\\Credit: Hasbro
\end{frame}

\begin{frame}
	\frametitle{Why Test?}
	\begin{itemize}
	\item Writing automated unit tests allows us to change code and ensure we do not break existing functionality. \pause
	\item Unit tests are a contract for the functionality of code. They communicate the intent of the code. \pause
	\item Unit tests act as documentation for code.
	\end{itemize}
\end{frame}

\begin{frame}
	\frametitle{Benefits \& Myths...}
\textbf{Benefits of testing:}
	\begin{itemize}
		\item Promotes code reuse.
		\item Less bugs! A test suite that evolves as you fix bugs so they NEVER occur again.
		\item Documents the code.
		\item Forces you to write more modular code.
		\end{itemize}\pause
		\textbf{Myths about testing:}
		\begin{itemize}
		\item Testing is hard.
		\item Testing takes too much time.
		\item Writing tests is only if you want it to be "perfect".
		\end{itemize}

\end{frame}

\begin{frame}
	\frametitle{Does this code need tests?}
	
	\begin{itemize}
\item	I have a piece of code that runs on a server every day at 9am.\\ 
	
\item 	It's run for 15 years and has not been altered since 1992.\\
	
\item	Should I go and write tests for this function?
\end{itemize}
\end{frame}

\begin{frame}
	\frametitle{Which of these need tests?}
	
	\begin{itemize}
		\item Simple 5 line script that calls an API \& sends an email.
		\item  Public function of a library that is in active development.
		\item Private function of a library that is in active development.
	\end{itemize}
\end{frame}

\begin{frame}
	\frametitle{Examples}
	Refer to example 2 in the Jupyter notebook.
\end{frame}


\section{Arrange-Act-Assert?}
\begin{frame}
	\frametitle{Arrange-Act-Assert}
	\begin{itemize}
		\item \textbf{Arrange:} Create inputs to the function or class you are testing.
		\item \textbf{Act:} Call the function or class you are testing.
		\item \textbf{Assert:} Assert that you get the output you expected.
	\end{itemize}
\end{frame}


\section{Writing Testable Code}
\begin{frame}
	\frametitle{What is Testable Code?}
	
	\begin{itemize}
		\item Deterministic. \pause
		\item In a function or class. \pause
		\item De-coupled.
	\end{itemize}
	
\end{frame}

\begin{frame}
	\frametitle{Refactoring Example}
	See Example 3 in the Jupyter notebook.
\end{frame}


\end{document}
